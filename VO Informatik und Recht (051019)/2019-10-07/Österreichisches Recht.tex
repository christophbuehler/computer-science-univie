\documentclass[12pt]{article}
\begin{document}
    \section{Österreichisches Recht}
    Das Öst. Recht besteht aus Gesetzen und Judikatur, der Umsetzung der Gesetze.

    Recht
    \begin{itemize}
        \item Zivilrecht (zwischen zwei vor dem Gesetz gleichen natürliche Personen).
        \begin{itemize}
            \item OSH
        \end{itemize}
        \item Strafrecht (Das Gegenüber ist der Staat. Rechtswidriges Verhalten.)
        \begin{itemize}
            \item OSH
        \end{itemize}
        \item Öffentliches Recht (Hierarchische Beziehung, Baurecht, Naturschutz, Gewerberecht)
        \begin{itemize}
            \item VwGH oder manchmal VfGH (wenn Verfassungsrecht betroffen) 
        \end{itemize}
    \end{itemize}

    \begin{itemize}
        \item Gesetze
        \begin{itemize}
            \item Bundesrecht
            \item Landesrecht
        \end{itemize}
        \item Judikatur
    \end{itemize}

    \subsection{Entstehung von Gesetzen}

    \begin{itemize}
        \item Ministerium macht Vorschlag.
        \begin{itemize}
            \item Ministerium bringt es zur Regierung.
            \item Regierung diskutiert \(\rightarrow\) Einstimmigkeitsgrundsatz.
            \item Regierungsvorschlag \(\rightarrow\) Nationalrat (183 Abgeordnete, Parlament).
            \item Wird diskutiert \(\rightarrow\) Ergebnis muss einfache Mehrheit sein.
            \item Gesetz kommt meist in den Bundesrat (Vertreter der Länder, föderaler Staat).
            \item Gesetz wird unterschrieben von Kanzler und Bundespräsident und im Bundesgesetzblatt kundgemacht.
        \end{itemize}
        \item Abgeordnete machen Vorschlag.
        \item Volk macht Vorschlag (Vorlksbegehren).
        \item Bundesrat macht Vorschlag.
    \end{itemize}

    \subsection{Judikatur}
    \begin{itemize}
        \item Was auf Grundlage der Gesetze entschieden wird
        \item Anwendung der Gesetze
        \item Erfolgt durch 3 Höchstgerichte
        \begin{itemize}
            \item VFGH (Verfassungsgerichtshof)
            \item VwGH (Verwaltungsgerichtshof)
            \item OGH (oberster Gerichtshof)
        \end{itemize}
    \end{itemize}
\end{document}