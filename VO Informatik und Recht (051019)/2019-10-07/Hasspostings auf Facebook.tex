\documentclass[12pt]{article}
\begin{document}
    \begin{center}
        \section*{Hasspostings auf Facebook}
    \end{center}
    \subsection{Sachverhalt}
    Politikerin wurde auf Facebook beileidigt.
    
    Rechtsanspruch des Opfers:
    \begin{itemize}
        \item Gegen Täter
        \item Aber auch gegen Facebook (Provider)
        \begin{itemize}
            \item Facebook sollte Posting löschen.
            \item Äusserung sollte weltweit gelöscht werden.
            \item \textbf{Inhaltsgleiche} Äusserungen sollten ebenfalls verhindert werden.
        \end{itemize}
    \end{itemize}

    \subsection{Rechte des Opfers}
    \begin{itemize}
        \item Menschliche Würde
    \end{itemize}

    \subsection{Rechte des Täters}
    \begin{itemize}
        \item Meinungsfreiheit
    \end{itemize}

    \subsection{Rechte von Facebook}
    \begin{itemize}
        \item Meinungsfreiheit
        \item Eigentum
    \end{itemize}

    \subsection{Ist der Rechtsanspruch gerechtfertigt?}
    Rechtsinformationssystem des Bundes:
    \begin{itemize}
        \item https://www.ris.bka.gv.at
        \item Stammt vom Justizministerium.
        \item URL noch alt, da sie im Gesetz steht!
    \end{itemize}

    Europäische Grundrechte Charta (seit 2010)
    \begin{itemize}
        \item Kapitel 8: Schutz personenbezogener Daten
        \item Grundrechte: EU, Europarat, Staatsgrundgesetz
    \end{itemize}

    E-Commerce-Gesetz
    \begin{itemize}
        \item Paragraph 0? \(\rightarrow\) Wie das Gesetz entstanden ist (Regelungsgeschichte).
    \end{itemize}

    Zivilrecht ist immer Bundesrecht in Österreich.
    OGH beantwortet Frage der Auslegung des europäischen Rechts.
    OGH musste E-Commerce-Gesetz anwenden
    
    Antworten von 3. Kammer
    \begin{itemize}
        \item Es gibt keine allgemeine Überwachungspflicht
        \item OGH sagt nicht, wie das Verfahren ausgehen muss
    \end{itemize}

    Es ist technisch nicht möglich.
    Fall noch nicht entschieden.

    \subsection{Territoriale Anwendbarkeit des Öst. Rechts}
    \begin{itemize}
        \item Internet: Hat das Unternehmen eine Niederlassung in Europa
        \item Facebook in Irland
        \item Auswirkungen weltweit, da Postings überall gelöscht
    \end{itemize}
\end{document}
