\documentclass[12pt]{article}
\begin{document}
    \section*{Mathematische Aussage}
    Besteht aus einer Voraussetzung p und einer Behauptung q.
    \[ p \Rightarrow q \] oder \[ p \Leftrightarrow q \]
    \section*{Mathematischer Beweis}
    Beweis, dass der logische Ausdruck eines mathematischen Satzes eine Tautologie ist (=Aussage, die immer wahr ist).

    \section*{Kommutativgesetz}
    Argumente einer Operation können vertauscht werden.

    \section*{Assoziativgesetz}
    Wenn die Reihenfolge keine Rolle spielt.

    \section*{Distributivgesetz}
    Wie sich eine Formel bei Auflösung der Klammern verhält.
\end{document}
