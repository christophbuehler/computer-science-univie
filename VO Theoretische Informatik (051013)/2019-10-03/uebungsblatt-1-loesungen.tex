\documentclass[12pt]{article}
\usepackage[utf8]{inputenc}
\usepackage{tikz}
\usepackage{enumitem}
\usepackage{mathtools}

\title{THI Übungsblatt 1 [Lösungen]}
\author{Christoph Bühler}
\date{\today}

\begin{document}
  \maketitle

  \section*{Aufgabe 1:}
  In dieser Aufgabe betrachten wir aussagenlogische Formeln über die atomaren Formeln
  \(Var = \{ A, B, C, D \}\). Beachten Sie dabei die formale Definition der Syntax der Aussagenlogik aus der Vorlesung.

    \begin{enumerate}[label=(\alph*)]
      \item Entscheiden Sie ob die folgenden Formeln syntaktisch korrekte aussagenlogische Formeln für die atomaren Formeln \(Var = \{ A, B, C, D \}\) sind und begründen Sie ihre Hypothese.

        \begin{enumerate}[label=(\roman*)]
          \item \((((A \wedge \neg B) \vee \neg D) \vee C)\)
          \item \((A \wedge C \vee (\neg \wedge BD))\)
          \item \((A \rightarrow B \rightarrow (C \wedge D))\)
          \item \((A \vee (B \rightarrow \neg (D \vee C)))\)
        \end{enumerate}
    \end{enumerate}

  \section*{Aufgabe 2:}
  ...
  \section*{Aufgabe 3:}
  ...
  \section*{Aufgabe 4:}
  Beweisen Sie (i) mittels Wahrheitstafel und (ii) mittels Umformungen auf Basis des Ersetzungssatzes und der in der Vorlesung präsentieren Äquivalenzen, dass die folgenden
Formeln semantisch äquivalent sind:

  \begin{center}
    \(\overbrace{A \wedge (C \vee D \vee \neg E)}^{F}\)\par
    und\par
    \(\underbrace{(A \wedge C) \vee (A \wedge D) \vee (A \wedge \neg E)}_{G}\)\par

  \end{center}

  \begin{enumerate}[label=(\roman*)]
    \item Wahrheitstafel
    \begin{center}
      \begin{tabular}{ |c c c c|c|c c c c| } 
        \hline
        A & C & D & E & \textbf{F} & \(A \wedge C\) & \(A \wedge D\) & \(A \wedge \neg E\) & \textbf{G} \\
        \hline
        0 & 0 & 0 & 0 & 0 & 0 & 0 & 0 & 0 \\
        0 & 0 & 0 & 1 & 0 & 0 & 0 & 0 & 0 \\
        0 & 0 & 1 & 0 & 0 & 0 & 0 & 0 & 0 \\
        0 & 0 & 1 & 1 & 0 & 0 & 0 & 0 & 0 \\
        0 & 1 & 0 & 0 & 0 & 0 & 0 & 0 & 0 \\
        0 & 1 & 0 & 1 & 0 & 0 & 0 & 0 & 0 \\
        0 & 1 & 1 & 0 & 0 & 0 & 0 & 0 & 0 \\
        0 & 1 & 1 & 1 & 0 & 0 & 0 & 0 & 0 \\
        1 & 0 & 0 & 0 & \textbf{1} & 0 & 0 & 1 & \textbf{1} \\
        1 & 0 & 0 & 1 & 0 & 0 & 0 & 0 & 0 \\
        1 & 0 & 1 & 0 & \textbf{1} & 0 & 1 & 1 & \textbf{1} \\
        1 & 0 & 1 & 1 & \textbf{1} & 0 & 1 & 0 & \textbf{1} \\
        1 & 1 & 0 & 0 & \textbf{1} & 1 & 0 & 1 & \textbf{1} \\
        1 & 1 & 0 & 1 & \textbf{1} & 1 & 0 & 0 & \textbf{1} \\
        1 & 1 & 1 & 0 & \textbf{1} & 1 & 1 & 1 & \textbf{1} \\
        1 & 1 & 1 & 1 & \textbf{1} & 1 & 1 & 0 & \textbf{1} \\
        \hline
      \end{tabular}
    \end{center}
    \(2^4 = 16\) Möglichkeiten für Argumente.

    \item Umformungen\par
    \((A \wedge C) \vee (A \wedge D) \vee (A \wedge \neg E) \rightarrow A \wedge (C \vee D \vee \neg E)\)\par
    
  \end{enumerate}

  Welche Methode erscheint Ihnen besser geeignet?\par
  \textbf{Die zweite Methode.}

  \section*{Aufgabe 5:}
  Ist die folgende Aussagenmenge widerspruchsfrei?
  \begin{enumerate}
    \item Wenn ein Bier zu lange offen steht, dann trinke ich es nicht.\par
    "Ein Bier steht zu lange offen." \(\rightarrow\) "Ich trinke es nicht."

    \item Wenn ich kein Bier trinke, dann kann ich Autofahren.\par
    "Ich trinke kein Bier." \(\rightarrow\) "Ich kann Autofahren."

    \item Ich kann nicht Autofahren.\par
    "Ich kann nicht Autofahren."

    \item Das Bier war zu lange offen.\par
    "Das Bier war zu lange offen."

  \end{enumerate}

  \textbf{Die Aussagenmenge ist widerspruchsfrei.}

  \section*{Aufgabe 6:}
  Formaliseren Sie folgende Sätze mittels Aussagenlogik und überprüfen Sie, ob der Schluss von (1),(2),(3) auf (4) korrekt ist.
  \begin{enumerate}
    \item Wenn ich schlafe, dann träume ich.\par
    "Ich schlafe." \(\rightarrow\) "Ich träume."

    \item Wenn ich esse, dann schlafe ich nicht.\par
    "Ich esse." \(\rightarrow \neg\) "Ich schlafe."

    \item Ich träume nicht.\par
    "Ich träume nicht."

    \item Ich träume nicht und ich esse oder lerne.\par
    \(\neg\) "Ich träume." \(\wedge\) ("Ich esse." \(\vee\) "Ich lerne.")

  \end{enumerate}

  \textbf{Schluss: "Ich lerne."}
\end{document}
